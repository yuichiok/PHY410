\documentclass{article}
% Change "article" to "report" to get rid of page number on title page
\usepackage{amsmath,amsfonts,amsthm,amssymb}
\usepackage{setspace}
\usepackage{Tabbing}
\usepackage{fancyhdr}
\usepackage{lastpage}
\usepackage{extramarks}
\usepackage{chngpage}
\usepackage{soul,color}
\usepackage{graphicx,float,wrapfig}
\usepackage{cancel}
\usepackage{relsize}
\usepackage{graphicx} 
\usepackage{multicol}
\usepackage{wrapfig}
\usepackage{tikz}
\usetikzlibrary{arrows,shapes.gates.logic.US,shapes.gates.logic.IEC,calc}
\usetikzlibrary{arrows.meta}
% In case you need to adjust margins:
\topmargin=-0.45in      %
\evensidemargin=0in     %
\oddsidemargin=0in      %
\textwidth=6.5in        %
\textheight=9.0in       %
\headsep=0.25in         %

% Homework Specific Information
\newcommand{\hmwkTitle}{Homework\ \#1}
\newcommand{\hmwkDueDate}{\today}
\newcommand{\hmwkClass}{PHY\ 410}
\newcommand{\hmwkClassTime}{4:00 pm}
\newcommand{\hmwkClassInstructor}{Professor Rappoccio}
\newcommand{\hmwkAuthorName}{Yuichi\ Okugawa\ }

% Setup the header and footer
\pagestyle{fancy}                                                       %
\lhead{\hmwkAuthorName}                                                 %
\chead{\hmwkClass\ (\hmwkClassInstructor\ \hmwkClassTime): \hmwkTitle}  %
\rhead{\firstxmark}                                                     %
\lfoot{\lastxmark}                                                      %
\cfoot{}                                                                %
\rfoot{Page\ \thepage\ of\ \pageref{LastPage}}                          %
\renewcommand\headrulewidth{0.4pt}                                      %
\renewcommand\footrulewidth{0.4pt}                                      %

% This is used to trace down (pin point) problems
% in latexing a document:
%\tracingall

%%%%%%%%%%%%%%%%%%%%%%%%%%%%%%%%%%%%%%%%%%%%%%%%%%%%%%%%%%%%%
%my tools
\newcommand{\rmd}{\mathrm{d}}
\newcommand{\Lagr}{\mathcal{L}}
\newcommand{\del}{\nabla}
\newcommand{\pfrac}[2]{\frac{\partial #1 }{\partial #2}}

% Some tools
\newcommand{\enterProblemHeader}[1]{\nobreak\extramarks{#1}{#1 continued on next page\ldots}\nobreak%
                                    \nobreak\extramarks{#1 (continued)}{#1 continued on next page\ldots}\nobreak}%
\newcommand{\exitProblemHeader}[1]{\nobreak\extramarks{#1 (continued)}{#1 continued on next page\ldots}\nobreak%
                                   \nobreak\extramarks{#1}{}\nobreak}%

\newlength{\labelLength}
\newcommand{\labelAnswer}[2]
  {\settowidth{\labelLength}{#1}%
   \addtolength{\labelLength}{0.25in}%
   \changetext{}{-\labelLength}{}{}{}%
   \noindent\fbox{\begin{minipage}[c]{\columnwidth}#2\end{minipage}}%
   \marginpar{\fbox{#1}}%

   % We put the blank space above in order to make sure this
   % \marginpar gets correctly placed.
   \changetext{}{+\labelLength}{}{}{}}%

\setcounter{secnumdepth}{0}
\newcommand{\homeworkProblemName}{}%
\newcounter{homeworkProblemCounter}%
\newenvironment{homeworkProblem}[1][Problem \arabic{homeworkProblemCounter}]%
  {\stepcounter{homeworkProblemCounter}%
   \renewcommand{\homeworkProblemName}{#1}%
   \section{\homeworkProblemName}%
   \enterProblemHeader{\homeworkProblemName}}%
  {\exitProblemHeader{\homeworkProblemName}}%

\newcommand{\problemAnswer}[1]
  {\noindent\fbox{\begin{minipage}[c]{\columnwidth}#1\end{minipage}}}%

\newcommand{\problemLAnswer}[1]
  {\labelAnswer{\homeworkProblemName}{#1}}

\newcommand{\homeworkSectionName}{}%
\newlength{\homeworkSectionLabelLength}{}%
\newenvironment{homeworkSection}[1]%
  {% We put this space here to make sure we're not connected to the above.
   % Otherwise the changetext can do funny things to the other margin

   \renewcommand{\homeworkSectionName}{#1}%
   \settowidth{\homeworkSectionLabelLength}{\homeworkSectionName}%
   \addtolength{\homeworkSectionLabelLength}{0.25in}%
   \changetext{}{-\homeworkSectionLabelLength}{}{}{}%
   \subsection{\homeworkSectionName}%
   \enterProblemHeader{\homeworkProblemName\ [\homeworkSectionName]}}%
  {\enterProblemHeader{\homeworkProblemName}%

   \changetext{}{+\homeworkSectionLabelLength}{}{}{}}%

\newcommand{\sectionAnswer}[1]
  {% We put this space here to make sure we're disconnected from the previous
   % passage

   \noindent\fbox{\begin{minipage}[c]{\columnwidth}#1\end{minipage}}%
   \enterProblemHeader{\homeworkProblemName}\exitProblemHeader{\homeworkProblemName}%
   \marginpar{\fbox{\homeworkSectionName}}%
   }%

%%%%%%%%%%%%%%%%%%%%%%%%%%%%%%%%%%%%%%%%%%%%%%%%%%%%%%%%%%%%%


%%%%%%%%%%%%%%%%%%%%%%%%%%%%%%%%%%%%%%%%%%%%%%%%%%%%%%%%%%%%%
% Make title
\title{\vspace{2in}\textmd{\textbf{\hmwkClass:\ \hmwkTitle}}\\\normalsize\vspace{0.1in}\small{Due\ on\ \hmwkDueDate}\\\vspace{0.1in}\large{\textit{\hmwkClassInstructor\ \hmwkClassTime}}\vspace{3in}}
\date{}
\author{\textbf{\hmwkAuthorName}}
%%%%%%%%%%%%%%%%%%%%%%%%%%%%%%%%%%%%%%%%%%%%%%%%%%%%%%%%%%%%%

\newcommand*\circled[1]{\tikz[baseline=(char.base)]{
            \node[shape=circle,draw,inner sep=2pt] (char) {#1};}}

\begin{document}
\begin{spacing}{1.1}
\maketitle
\newpage
\tableofcontents
\newpage

\clearpage
%%%%%%%%%              Problem 1            %%%%%%%%%%%%%%%%%%%%%%%%%%%%%%%%%
\begin{homeworkProblem}[Problem 2]
What are the two's complement representations for the following (decimal) numbers? Show your work. Submit a tex file or equivalent (word, pages, etc) on your github directory ``Homework 1".\\
a) 10\\
b) 436\\
c) 1024\\
d) -13\\
e) -1023\\
f) -1024\\
\noindent\rule[0.15\baselineskip]{\textwidth}{2pt}
\noindent{\bf Solution\,}\\
a)  $10=1\cdot2^3+0\cdot2^2+1\cdot2^1+0\cdot2^0$.\\
Since it is a positive number, we add 0 in front to distinguish from negative ones. Therefore,\\
$$10=01010$$
\\b) Divide the number by two. If there is a remainder, we print 1, otherwise 0. We repeat this process until we reach 1. Below is the procedure.
$$
 \begin{array}{|l}
    \llap{2~~~~} 436 \cdots 0 \\ \hline
    \llap{2~~~~} 218 \cdots 0  \\ \hline
    \llap{2~~~~} 109 \cdots 1 \\ \hline
    \llap{2~~~~} 54   \cdots 0 \\ \hline
    \llap{2~~~~} 27   \cdots 1 \\ \hline
    \llap{2~~~~} 13   \cdots 1 \\ \hline
    \llap{2~~~~} 6     \cdots 0\\ \hline
    \llap{2~~~~} 3     \cdots 1 \\ \hline
                 1
\end{array}
$$
To convert 436 into binary, we simply read these remainders from the bottom. Therefore\\
$$436=0110110100$$
\\
c) It is useful to know that $1024=2^{10}$. Therefore,\\
$$1024=010000000000$$
\\
d) To find two's complement of negative decimal, we simply flip the bits, then add 1.
First we find binary expression of 13. $13=1\cdot2^3+1\cdot2^2+0\cdot2^1+1\cdot2^0$. So\\ 
$$13=01101$$
Now we flip bits.
$$01101\rightarrow10010$$
Add 1.
$$10010+00001=10011$$
\\
e) We will come back to this part later, I promise.
\\
\\f) Combined the binary form we found in part c and technique we used in part d, we can figure out $-1024$.
$$1024=010000000000$$
Now we flip bits.
$$010000000000\rightarrow101111111111$$
Add 1.
$$101111111111+000000000001=110000000000$$
\\
e) Now, all we have to do is to add 1 to the answer in part f since $-1023=-1024+1$.
$$-1023=110000000000+000000000001=110000000001$$
\qed



\end{homeworkProblem}
\pagebreak
\clearpage
\end{spacing}
\end{document}
%----------------------------------------------------------------------%
